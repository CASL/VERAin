%*-------------------------------------------------------------------------*
%*                 Copyright 2013-2015 Core Physics, Inc.                  *
%*  Under terms of the contract to support CASL, there is a non-exclusive  *
%*  license for use of this work by or on behalf of the U.S. Government.   *
%*-------------------------------------------------------------------------*
%
% WARNING: This file is autogenerated, do not modify by hand!
%

%%%%%%%%%%%%%%%%%%%%%%%%%%%%%%%%%%%%%%%%%%%%%%%
\section{Block CASEID}
\label{sec:caseidcards}

{\bf [CASEID]}

{\bf title} case\_id
\begin{cardlist}
  \item[case\_id]  Problem name
\end{cardlist}

%%%%%%%%%%%%%%%%%%%%%%%%%%%%%%%%%%%%%%%%%%%%%%%
\section{Block STATE}
\label{sec:statecards}

{\bf [STATE]}

{\bf title} title
\begin{cardlist}
  \item[title]  Statepoint title
\end{cardlist}

{\bf op\_date} op\_date
\begin{cardlist}
  \item[op\_date]  Operating date of this statepoint.  Used when writing restart files. (``MM/DD/YYYY'' or ``YYYY/MM/DD'')
\end{cardlist}

{\bf power} power
\begin{cardlist}
  \item[power]  Operating power (percent of rated power)
\end{cardlist}

{\bf flow} flow
\begin{cardlist}
  \item[flow]   Operating flow  (percent of rated flow)
\end{cardlist}

{\bf bypass} bypass
\begin{cardlist}
  \item[bypass]  Bypass flow (percent of rated flow)
\end{cardlist}

{\bf tinlet} tinlet units
\begin{cardlist}
  \item[tinlet]  Inlet temperature
  \item[units]  (``F'', ``K'', or ``C'')
\end{cardlist}

{\bf tfuel} tfuel units
\begin{cardlist}
  \item[tfuel]  Fixed fuel temperatures - Only used if feedback is turned OFF
  \item[units]  (``F'', ``K'', or ``C'')
\end{cardlist}

{\bf modden} modden
\begin{cardlist}
  \item[modden]  Fixed moderator density (g/cc) - Only used if feedback is turned OFF
\end{cardlist}

{\bf xenon} xenopt
\begin{cardlist}
  \item[xenopt]  Xenon option (``zero'', ``dep'', or ``equil'')
\end{cardlist}

{\bf samar} samopt
\begin{cardlist}
  \item[samopt]  Samarium option (``zero'', ``dep'', ``equil'', or ``peak'')
\end{cardlist}

{\bf boron} boron
\begin{cardlist}
  \item[boron]  Soluble boron concentration (ppm)
\end{cardlist}

{\bf b10} b10 b10\_depl
\begin{cardlist}
  \item[b10]  Boron-10 fraction in coolant (atom percent) (default is 19.9 atom percent)
  \item[b10\_depl]  Flag to enable B-10 depletion in coolant (``on'' or ``off'', default is ``off'')
\end{cardlist}

{\bf kcrit} kcrit
\begin{cardlist}
  \item[kcrit]  Critical eigenvalue used in boron search
\end{cardlist}

{\bf search} search
\begin{cardlist}
  \item[search]  Search option (``keff'' or ``boron'')
\end{cardlist}

{\bf pressure} pressure
\begin{cardlist}
  \item[pressure]  Core exit pressure (psia)
\end{cardlist}

{\bf deplete} deplete\_units deplete
\begin{cardlist}
  \item[deplete\_units]  Depletion units (``EFPD'', ``GWDMT'', ``hours'')
  \item[deplete]  List of depletion exposure steps - only used with depletion
\end{cardlist}

{\bf rodbank} bank\_pos bank\_labels
\begin{cardlist}
  \item[bank\_pos]  Steps withdrawn for each bank in list
  \item[bank\_labels]  List of control rod banks to position.  Labels correspond to crd\_map in CORE block.
\end{cardlist}

{\bf feedback} feedback
\begin{cardlist}
  \item[feedback]  Flag to turn on and off T/H feedback (``on'' or ``off'').  Currently not used, feedback is controlled by executable that is run.
\end{cardlist}

{\bf sym} sym
\begin{cardlist}
  \item[sym]  Core fraction to run problem in (``full'' or ``qtr'')
\end{cardlist}

{\bf restart\_shuffle} restart\_shuffle\_file restart\_shuffle\_label
\begin{cardlist}
  \item[restart\_shuffle\_file]  List of restart files to search during core shuffle
  \item[restart\_shuffle\_label]  List of user labels on restart files to search during core shuffle
\end{cardlist}

{\bf restart\_read} restart\_read\_file restart\_read\_label
\begin{cardlist}
  \item[restart\_read\_file]  Name of restart file to read
  \item[restart\_read\_label]  User label on restart file to read
\end{cardlist}

{\bf restart\_write} restart\_write\_file restart\_write\_label
\begin{cardlist}
  \item[restart\_write\_file]  Name of restart file to write/create
  \item[restart\_write\_label]  User label to use when writing restart file
\end{cardlist}

{\bf shuffle\_label} shuffle\_label
\begin{cardlist}
  \item[shuffle\_label]  Core map showing core shuffle instructions
\end{cardlist}

%%%%%%%%%%%%%%%%%%%%%%%%%%%%%%%%%%%%%%%%%%%%%%%
\section{Block CORE}
\label{sec:corecards}

{\bf [CORE]}

{\bf name} core\_name
\begin{cardlist}
  \item[core\_name]  Name of the reactor core
\end{cardlist}

{\bf cycle} cycle\_num
\begin{cardlist}
  \item[cycle\_num]  Cycle number (string)
\end{cardlist}

{\bf unit} unit
\begin{cardlist}
  \item[unit]   Reactor plant unit name. Only used for multi-unit sites with cross-unit shuffle. (string)
\end{cardlist}

{\bf op\_date} op\_date
\begin{cardlist}
  \item[op\_date]  Start-up date of core reload.  Only used when performing core shuffle. (``MM/DD/YYYY'' or ``YYYY/MM/DD'')
\end{cardlist}

{\bf size} core\_size
\begin{cardlist}
  \item[core\_size]  Number of assemblies across one axis in full-core geometry (required)
\end{cardlist}

{\bf rated} rated\_power rated\_flow
\begin{cardlist}
  \item[rated\_power]  Rated thermal power at 100\% power (MW)
  \item[rated\_flow]   Rated vessel flow at 100\% flow (Mlbs/hr)
\end{cardlist}

{\bf rcs\_volume} rcs\_volume
\begin{cardlist}
  \item[rcs\_volume]  Volume of the Reactor Coolant System (cubic ft) (only used with B-10 depletion)
\end{cardlist}

{\bf apitch} apitch
\begin{cardlist}
  \item[apitch]  Assembly pitch (cm)
\end{cardlist}

{\bf baffle} baffle\_mat baffle\_gap baffle\_thick
\begin{cardlist}
  \item[baffle\_mat]   Baffle material
  \item[baffle\_gap]   Gap between outside assembly (including assembly gap) and baffle (cm)
  \item[baffle\_thick]   Thickness of baffle (cm)
\end{cardlist}

{\bf vessel} vessel\_mats vessel\_radii
\begin{cardlist}
  \item[vessel\_mats]  Vessel materials
  \item[vessel\_radii]  Vessel radii (cm)
\end{cardlist}

{\bf core\_shape} shape
\begin{cardlist}
  \item[shape]  Square map showing the fuel assembly locations.  Enter 1 for fuel assembly locations and 0 for empty locations.
\end{cardlist}

{\bf rotate\_map} rotate\_map
\begin{cardlist}
  \item[rotate\_map]  Core map of assembly rotations (0-3)
\end{cardlist}

{\bf assm\_map} assm\_map
\begin{cardlist}
  \item[assm\_map]  Core map of the fuel assembly types.  The assembly types correspond to assembly labels in the ASSEMBLY block.  All fuel assemblies must have a type defined.
\end{cardlist}

{\bf insert\_map} insert\_map
\begin{cardlist}
  \item[insert\_map]  Core map of the fuel insert types and locations.  The insert types correspond to insert labels in the INSERT block.  Use a dash to specify assemblies with no inserts.
\end{cardlist}

{\bf det\_map} det\_map
\begin{cardlist}
  \item[det\_map]  Core map of the detector types and locations.  The detector types correspond to detector labels in the DETECTOR block.  Use a dash to specify assemblies with no detectors.
\end{cardlist}

{\bf crd\_map} crd\_map
\begin{cardlist}
  \item[crd\_map]  Core map of the control rod types and locations.  The control rod types correspond to control rod labels in the CONTROL block.  Use a dash to specify assemblies with no control rods.
\end{cardlist}

{\bf crd\_bank} crd\_bank
\begin{cardlist}
  \item[crd\_bank]  Core map of the control rod bank labels.  These labels are used to position groups of control rods by bank label.  Use a dash to specify assemblies with no control rods.
\end{cardlist}

{\bf lower\_plate} lower\_mat lower\_thick lower\_vfrac
\begin{cardlist}
  \item[lower\_mat]  Lower core plate material
  \item[lower\_thick]  Lower core plate thickness (cm)
  \item[lower\_vfrac]  Lower core plate material volume fraction.  Remainder of volume fraction will be filled with coolant.
\end{cardlist}

{\bf upper\_plate} upper\_mat upper\_thick upper\_vfrac
\begin{cardlist}
  \item[upper\_mat]  Upper core plate material
  \item[upper\_thick]  Upper core plate thickness (cm)
  \item[upper\_vfrac]  Upper core plate material volume fraction.  Remainder of volume fraction will be filled with coolant.
\end{cardlist}

{\bf bc\_sym} bc\_sym
\begin{cardlist}
  \item[bc\_sym]  Symmetry flag for the core when using qtr-symmetry.  Flag is not used in full-symmetry.  Valid  options are ``rot'' and ``mir''.
\end{cardlist}

{\bf bc\_bot} bc\_bot
\begin{cardlist}
  \item[bc\_bot]  Bottom neutron transport boundary condition, ``vacuum'' (default) or ``reflecting''.
\end{cardlist}

{\bf bc\_top} bc\_top
\begin{cardlist}
  \item[bc\_top]  Top neutron transport boundary condition, ``vacuum'' (default) or ``reflecting''.
\end{cardlist}

{\bf bc\_rad} bc\_rad
\begin{cardlist}
  \item[bc\_rad]  Radial neutron transport boundary condition, ``vacuum'' (default) or ``reflecting''.
\end{cardlist}

{\bf xlabel} xlabel
\begin{cardlist}
  \item[xlabel]  List of 2-character assembly position labels in x-direction.  These values are used in the edit maps.
\end{cardlist}

{\bf ylabel} ylabel
\begin{cardlist}
  \item[ylabel]  List of 2-character assembly position labels in y-direction.  These values are used in the edit maps.
\end{cardlist}

{\bf height} height
\begin{cardlist}
  \item[height]  Total axial distance from bottom core plate to upper core plate (cm).  Distance does not include core plate thicknesses.
\end{cardlist}

{\bf mat} mat
\begin{cardlist}
  \item[mat]  Refer to the detailed materials description given in the User's Manual.
\end{cardlist}

{\bf lower\_ref} lower\_refl\_mats lower\_refl\_thicks lower\_refl\_vfracs
\begin{cardlist}
  \item[lower\_refl\_mats]  Lower reflector materials
  \item[lower\_refl\_thicks]  Lower reflector thicknesses (cm)
  \item[lower\_refl\_vfracs]  Lower reflector volume fractions
\end{cardlist}

{\bf upper\_ref} upper\_refl\_mats upper\_refl\_thicks upper\_refl\_vfracs
\begin{cardlist}
  \item[upper\_refl\_mats]  Upper reflector materials
  \item[upper\_refl\_thicks]  Upper reflector thicknesses (cm)
  \item[upper\_refl\_vfracs]  Upper reflector volume fractions
\end{cardlist}

%%%%%%%%%%%%%%%%%%%%%%%%%%%%%%%%%%%%%%%%%%%%%%%
\section{Block ASSEMBLY}
\label{sec:assemblycards}

{\bf [ASSEMBLY]}

{\bf title} title
\begin{cardlist}
  \item[title]  Long descriptive title for assembly.
\end{cardlist}

{\bf npin} num\_pins
\begin{cardlist}
  \item[num\_pins]  The number of rods along the edge of an assembly.
\end{cardlist}

{\bf ppitch} ppitch
\begin{cardlist}
  \item[ppitch]  Pincell pitch (cm)
\end{cardlist}

{\bf cell} cell
\begin{cardlist}
  \item[cell]  Refer to the cell description given in the User's Manual.
\end{cardlist}

{\bf rodmap} axial\_label cell\_map
\begin{cardlist}
  \item[axial\_label]  Label for this axial elevation description.
  \item[cell\_map]  Lattice map for this axial elevation.  Use a dash for an empty location.
\end{cardlist}

{\bf axial} label axial\_labels axial\_elevations
\begin{cardlist}
  \item[label]  Label for this assembly.  Label corresponds to assm\_map in CORE block.
  \item[axial\_labels]  List of axial labels for this assembly description.  Corrrespond to labels in lattice maps.
  \item[axial\_elevations]  List of axial elevations for this assembly description (cm).
\end{cardlist}

{\bf dancoff} dancoff\_map
\begin{cardlist}
  \item[dancoff\_map]  Lattice map of Dancoff factors (not currently used).
\end{cardlist}

{\bf grid} label material mass height
\begin{cardlist}
  \item[label]  Grid label for a single grid type.
  \item[material]  Grid material for this grid type.
  \item[mass]      Grid mass for this grid type (g).
  \item[height]    Grid height for this grid type (cm).
\end{cardlist}

{\bf grid\_axial} grid\_map grid\_elev
\begin{cardlist}
  \item[grid\_map]   List of spacer grid labels for all grids in an assembly (labels correspond to grid card)
  \item[grid\_elev]  List of spacer grid elevations for all grids in an assembly (cm).  Elevations refer to the grid midpoint.
\end{cardlist}

{\bf lower\_nozzle} lower\_nozzle\_comp lower\_nozzle\_height lower\_nozzle\_mass
\begin{cardlist}
  \item[lower\_nozzle\_comp]  Lower nozzle material.
  \item[lower\_nozzle\_height]  Lower nozzle height (cm)
  \item[lower\_nozzle\_mass]  Lower nozzle mass (g).  Code will calculate the volume of the nozzle given the nozzle mass, and use coolant for remaining volume.
\end{cardlist}

{\bf upper\_nozzle} upper\_nozzle\_comp upper\_nozzle\_height upper\_nozzle\_mass
\begin{cardlist}
  \item[upper\_nozzle\_comp]  Upper nozzle material.
  \item[upper\_nozzle\_height]  Upper nozzle height (cm)
  \item[upper\_nozzle\_mass]  Upper nozzle mass (g).  Code will calculate the volume of the nozzle given the nozzle mass, and use coolant for remaining volume.
\end{cardlist}

{\bf fuel} fuel
\begin{cardlist}
  \item[fuel]  Refer to the detailed materials description given in the User's Manual.
\end{cardlist}

{\bf mat} mat
\begin{cardlist}
  \item[mat]  Refer to the detailed materials description given in the User's Manual.
\end{cardlist}

%%%%%%%%%%%%%%%%%%%%%%%%%%%%%%%%%%%%%%%%%%%%%%%
\section{Block CONTROL}
\label{sec:controlcards}

{\bf [CONTROL]}

{\bf title} title
\begin{cardlist}
  \item[title]  Long descriptive title for control rod description.
\end{cardlist}

{\bf npin} num\_pins
\begin{cardlist}
  \item[num\_pins]  The number of rods along the edge of an assembly.
\end{cardlist}

{\bf stroke} stroke maxstep
\begin{cardlist}
  \item[stroke]  Control rod stroke - distance between full-insertion and full-withdrawal (cm)
  \item[maxstep]  Total number of steps between full-insertion and full-withdrawal
\end{cardlist}

{\bf cell} Cell
\begin{cardlist}
  \item[Cell]   Refer to the cell description given in the User's Manual.
\end{cardlist}

{\bf rodmap} label cell\_map
\begin{cardlist}
  \item[label]  Label for this axial elevation description
  \item[cell\_map]  Lattice map for this axial elevation.  Use a dash for no control rod.
\end{cardlist}

{\bf axial} control\_label axial\_labels axial\_elevations
\begin{cardlist}
  \item[control\_label]  Label for this control rod description.  Label corresponds to crd\_map in CORE block.
  \item[axial\_labels]  List of axial labels for this control rod description.  Corrrespond to labels in rod maps.
  \item[axial\_elevations]  List of axial elevations for this control rod description (cm).
\end{cardlist}

{\bf mat} mat
\begin{cardlist}
  \item[mat]  Refer to the detailed materials description given in the User's Manual.
\end{cardlist}

%%%%%%%%%%%%%%%%%%%%%%%%%%%%%%%%%%%%%%%%%%%%%%%
\section{Block INSERT}
\label{sec:insertcards}

{\bf [INSERT]}

{\bf title} title
\begin{cardlist}
  \item[title]  Long descriptive title for assembly insert description.
\end{cardlist}

{\bf npin} num\_pins
\begin{cardlist}
  \item[num\_pins]  The number of rods along the edge of an assembly.
\end{cardlist}

{\bf cell} Cell
\begin{cardlist}
  \item[Cell]   Refer to the cell description given in the User's Manual.
\end{cardlist}

{\bf rodmap} label cell\_map
\begin{cardlist}
  \item[label]  Label for this axial elevation description
  \item[cell\_map]  Lattice map for this axial elevation. Use a dash for no insert rod.
\end{cardlist}

{\bf axial} insert\_label axial\_labels axial\_elevations
\begin{cardlist}
  \item[insert\_label]  Label for this assembly insert description.  Label corresponds to insert\_map in CORE block.
  \item[axial\_labels]  List of axial labels for this assembly insert description.  Corrrespond to labels in rod maps.
  \item[axial\_elevations]  List of axial elevations for this assembly insert description (cm).
\end{cardlist}

{\bf mat} mat
\begin{cardlist}
  \item[mat]  Refer to the detailed materials description given in the User's Manual.
\end{cardlist}

%%%%%%%%%%%%%%%%%%%%%%%%%%%%%%%%%%%%%%%%%%%%%%%
\section{Block DETECTOR}
\label{sec:detectorcards}

{\bf [DETECTOR]}

{\bf title} title
\begin{cardlist}
  \item[title]  Long descriptive title for detector description.
\end{cardlist}

{\bf npin} num\_pins
\begin{cardlist}
  \item[num\_pins]  The number of rods along the edge of an assembly.
\end{cardlist}

{\bf cell} Cell
\begin{cardlist}
  \item[Cell]   Refer to the cell description given in the User's Manual.
\end{cardlist}

{\bf rodmap} label cell\_map
\begin{cardlist}
  \item[label]  Label for this axial elevation description
  \item[cell\_map]  Lattice map for this axial elevation. Use a dash for no detector rod.
\end{cardlist}

{\bf axial} detector\_label axial\_labels axial\_elevations
\begin{cardlist}
  \item[detector\_label]  Label for this detector description.  Label corresponds to det\_map in CORE block.
  \item[axial\_labels]  List of axial labels for this detector description.  Corrrespond to labels in rod maps.
  \item[axial\_elevations]  List of axial elevations for this detector description (cm).
\end{cardlist}

{\bf mat} mat
\begin{cardlist}
  \item[mat]  Refer to the detailed materials description given in the User's Manual.
\end{cardlist}

%%%%%%%%%%%%%%%%%%%%%%%%%%%%%%%%%%%%%%%%%%%%%%%
\section{Block EDITS}
\label{sec:editscards}

{\bf [EDITS]}

{\bf axial\_edit\_bounds} axial\_edit\_bounds
\begin{cardlist}
  \item[axial\_edit\_bounds]   The boundaries of the axial regions over which axial information should be printed.
\end{cardlist}

{\bf axial\_edit\_mesh\_delta} axial\_edit\_mesh\_delta
\begin{cardlist}
  \item[axial\_edit\_mesh\_delta]  Produces a uniform axial output grid (integrates pin powers over a uniform axial mesh).
\end{cardlist}

%%%%%%%%%%%%%%%%%%%%%%%%%%%%%%%%%%%%%%%%%%%%%%%
\section{Block COBRATF}
\label{sec:cobratfcards}

{\bf [COBRATF]}

{\bf nfuel} nfuel
\begin{cardlist}
  \item[nfuel]  Number of radial nodes in fuel pellet.
\end{cardlist}

{\bf nc} nc
\begin{cardlist}
  \item[nc]  Conduction option (see CTF Manual).
\end{cardlist}

{\bf debug} debug
\begin{cardlist}
  \item[debug]  Option to print additional debug information in xml2ctf processing  (1=on)
\end{cardlist}

{\bf irfc} irfc
\begin{cardlist}
  \item[irfc]  Friction factor correlation number (see CTF Manual).
\end{cardlist}

{\bf dhfrac} dhfrac
\begin{cardlist}
  \item[dhfrac]  Fraction of power deposited directly in coolant.
\end{cardlist}

{\bf hgap} hgap
\begin{cardlist}
  \item[hgap]  Gap conductance (W/m$^2$ K)
\end{cardlist}

{\bf gridloss} gridlabel gridloss
\begin{cardlist}
  \item[gridlabel]  Spacer grid label (from ASSEMBLY block).
  \item[gridloss]  Spacer grid loss coefficient.
\end{cardlist}

{\bf epso} epso
\begin{cardlist}
  \item[epso]  Inactive option - ignored by code.
\end{cardlist}

{\bf oitmax} oitmax
\begin{cardlist}
  \item[oitmax]  Inactive option - ignored by code
\end{cardlist}

{\bf iitmax} iitmax
\begin{cardlist}
  \item[iitmax]  Inactive option - ignored by code
\end{cardlist}

{\bf dtmin} dtmin
\begin{cardlist}
  \item[dtmin]  Minimum time step (s).
\end{cardlist}

{\bf dtmax} dtmax
\begin{cardlist}
  \item[dtmax]  Maximum time step (s).
\end{cardlist}

{\bf tend} tend
\begin{cardlist}
  \item[tend]  End of time domain (s).
\end{cardlist}

{\bf rtwfp} rtwfp
\begin{cardlist}
  \item[rtwfp]  Ratio of time step sizes for conduction/fluid.
\end{cardlist}

{\bf maxits} maxits
\begin{cardlist}
  \item[maxits]  Maximum number of iterations.
\end{cardlist}

{\bf courant} courant
\begin{cardlist}
  \item[courant]  Courant limit
\end{cardlist}

{\bf maps\_filename} maps\_filename
\begin{cardlist}
  \item[maps\_filename]   Name of HDF5 and VTK files
\end{cardlist}

{\bf heated\_elements\_type} heated\_elements\_type
\begin{cardlist}
  \item[heated\_elements\_type]  0=model nuclear fuel rods, 1=model electric heater tubes. If you select electric heater tubes, all heater rods in the model will be electric heater tubes and you must input the inside diameter, outside diameter, and pitch of the tubes.
\end{cardlist}

{\bf heater\_tube\_id} heater\_tube\_id
\begin{cardlist}
  \item[heater\_tube\_id]  Heater tube inside diameter (cm)
\end{cardlist}

{\bf heater\_tube\_od} heater\_tube\_od
\begin{cardlist}
  \item[heater\_tube\_od]  Heater tube outside diameter (cm)
\end{cardlist}

{\bf heater\_tube\_pitch} heater\_tube\_pitch
\begin{cardlist}
  \item[heater\_tube\_pitch]  Heater tube pitch (cm)
\end{cardlist}

{\bf solver} solver
\begin{cardlist}
  \item[solver]  Pressure matrix solver (See ISOL in CTF Manual)
\end{cardlist}

{\bf parallel} parallel
\begin{cardlist}
  \item[parallel]  0=serial execution, 1=parallel execution.  Parallelization is on a per-assembly basis.
\end{cardlist}

{\bf global\_energy\_balance} global\_energy\_balance
\begin{cardlist}
  \item[global\_energy\_balance]  Steady-state convergence criteria for balance of energy in the model
\end{cardlist}

{\bf global\_mass\_balance} global\_mass\_balance
\begin{cardlist}
  \item[global\_mass\_balance]  Steady-state convergence criteria for balance of mass in the model
\end{cardlist}

{\bf fluid\_energy\_storage} fluid\_energy\_storage
\begin{cardlist}
  \item[fluid\_energy\_storage]  Steady-state convergence criteria for transient storage of energy in fluid
\end{cardlist}

{\bf solid\_energy\_storage} solid\_energy\_storage
\begin{cardlist}
  \item[solid\_energy\_storage]  Steady-state convergence criteria for transient storage of energy in solids
\end{cardlist}

{\bf mass\_storage} mass\_storage
\begin{cardlist}
  \item[mass\_storage]  Steady-state convergence criteria for transient storage of mass in fluid
\end{cardlist}

{\bf edit\_gaps} edit\_gaps
\begin{cardlist}
  \item[edit\_gaps]  Edit flag to turn on gap output file (1=generate file)
\end{cardlist}

{\bf edit\_channels} edit\_channels
\begin{cardlist}
  \item[edit\_channels]  Edit flag to turn on channel output file (1=generate file)
\end{cardlist}

{\bf edit\_rods} edit\_rods
\begin{cardlist}
  \item[edit\_rods]  Edit flag to turn on rod edits (temperatures) in deck.out file (1=generate file)
\end{cardlist}

{\bf edit\_dnb} edit\_dnb
\begin{cardlist}
  \item[edit\_dnb]  Edit flag to turn on DNB output file (1=generate file)
\end{cardlist}

{\bf edit\_convergence} edit\_convergence
\begin{cardlist}
  \item[edit\_convergence]  Edit flag to generate a file displaying convergence parameters each iteration
\end{cardlist}

{\bf edit\_hdf5} edit\_hdf5
\begin{cardlist}
  \item[edit\_hdf5]  Option to generate the HDF5 file (1=generate file)
\end{cardlist}

{\bf boil\_ht\_cor} boil\_ht\_cor
\begin{cardlist}
  \item[boil\_ht\_cor]  Boiling heat transfer correlation
\end{cardlist}

{\bf property\_evaluations} property\_evaluations
\begin{cardlist}
  \item[property\_evaluations]  Source for fluid property evaluations (``asme'' for ASME-68, ``iapws'' for IAPWS-IF97)
\end{cardlist}

{\bf beta\_sp} beta\_sp
\begin{cardlist}
  \item[beta\_sp]  Turbulent mixing coefficient (used for both single-phase and two-phase mixing) (default=0.05)
\end{cardlist}

{\bf k\_void\_drift} k\_void\_drift
\begin{cardlist}
  \item[k\_void\_drift]   Equilibrium distribution weighting factor (void drift model coefficient) (default=1.4)
\end{cardlist}

%%%%%%%%%%%%%%%%%%%%%%%%%%%%%%%%%%%%%%%%%%%%%%%
\section{Block INSILICO}
\label{sec:insilicocards}

{\bf [INSILICO]}

Please refer to the Insilico documentation for a listing of all the cards in the [INSILICO] block.

%%%%%%%%%%%%%%%%%%%%%%%%%%%%%%%%%%%%%%%%%%%%%%%
\section{Block MPACT}
\label{sec:mpactcards}

{\bf [MPACT]}

Please refer to the MPACT documentation for a listing of all the cards in the [MPACT] block.

%%%%%%%%%%%%%%%%%%%%%%%%%%%%%%%%%%%%%%%%%%%%%%%
\section{Block COUPLING}
\label{sec:couplingcards}

{\bf [COUPLING]}

{\bf epsk} epsk
\begin{cardlist}
  \item[epsk]  Eigenvalue converence criteria (pcm)
\end{cardlist}

{\bf epsp} epsp
\begin{cardlist}
  \item[epsp]  Power convergence criteria (L2 norm)
\end{cardlist}

{\bf eps\_temp} eps\_temp
\begin{cardlist}
  \item[eps\_temp]  Temperature convergence criteria (degrees F)
\end{cardlist}

{\bf eps\_boron} eps\_boron
\begin{cardlist}
  \item[eps\_boron]  Boron convergence criteria (ppm)
\end{cardlist}

{\bf rlx\_power} rlx\_power
\begin{cardlist}
  \item[rlx\_power]  Power relaxation factor.  Recommend 0.5.
\end{cardlist}

{\bf rlx\_tfuel} rlx\_tfuel
\begin{cardlist}
  \item[rlx\_tfuel]  Fuel temperature relaxation factor.  Recommend 1.0.
\end{cardlist}

{\bf rlx\_den} rlx\_den
\begin{cardlist}
  \item[rlx\_den]  Density relaxation factor.  Recommend 1.0.
\end{cardlist}

{\bf maxiter} maxiter
\begin{cardlist}
  \item[maxiter]  Maximum number of coupled iterations.
\end{cardlist}

{\bf read\_restart} read\_restart
\begin{cardlist}
  \item[read\_restart]  Name of coupling restart file.  Leave blank for no coupling restart.
\end{cardlist}

{\bf ctf\_iters\_max} ctf\_iters\_max
\begin{cardlist}
  \item[ctf\_iters\_max]  Maximum number of CTF time-steps per coupled iteration
\end{cardlist}

{\bf ctf\_iters\_growth} ctf\_iters\_growth
\begin{cardlist}
  \item[ctf\_iters\_growth]  Fractional change in ctf\_iters\_max by coupled iteration (1 is no change)
\end{cardlist}
