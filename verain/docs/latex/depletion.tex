%*-------------------------------------------------------------------------*
%*                 Copyright 2013-2015 Core Physics, Inc.                  *
%*  Under terms of the contract to support CASL, there is a non-exclusive  *
%*  license for use of this work by or on behalf of the U.S. Government.   *
%*-------------------------------------------------------------------------*
%
% @version CVS $Id: depletion.tex,v 1.7 2015/02/23 02:31:24 scott Exp $
%
%%%%%%%%%%%%%%%%%%%%%%%%%%%%%%%%%%%%%%%%%
\chapter{Depletion}  \label{chap:depletion}

This chapter describes depletion and working with restart files.

Depletion and restart files are only available with MPACT.

%%%%%%%%%%%%%%%%%%%%%%%%%%%%%%%%
\section{Depletion}

Depletion refers to taking a step in time and calculating the change in number densities (isotopics) in the core.

A problem is depleted by including a {\it deplete} card in the [STATE] block,
as in the following example:
\begin{verbatim}
  [STATE]
    deplete EFPD 0.0 1.0 10.0 30.0
\end{verbatim}
The first parameter on the card is the units used in the depletion and can be ``EFPD'' for Effective Full-Power Days,
``GWDMT'' for Giga-Watt-days per metric ton of initial heavy metal, or ``hours''.  
Following the unit is a list of depletion steps to take.  
Each depletion step is referred to as a ``statepoint'' in the calculation.

Listing multiple statepoints on a single {\it deplete} card will deplete with all of the other
values in the [STATE] block held constant.  If the user wants to change a state parameter between
depletion steps (power, flow, etc.), they can split the depletion over multiple [STATE] blocks.
In the following example, the code depletes three statepoints at 50\% power, changes the power
to 100\%, and depletes for four more statepoints.  The statepoint at 10 EFPD is run
at both 50\% and 100\% power.
\begin{verbatim}
  [STATE]
    power 50.0
    deplete EFPD 0.0 1.0 10.0
  [STATE]
    power 100.0
    deplete EFPD 10.0 30.0 60.0 90.0
\end{verbatim}


%%%%%%%%%%%%%%%%%%%%%%%%%%%%%%%%
\section{Writing Restart Files}

Often a user will want to run a depletion and save the isotopic data to a file that can be used to 
restart a calculation at a later time.  This is useful if a calculation is long-running, and it needs to be
split into multiple cases.  Other times a user may want to save certain statepoints so they can go
back and run perturbation, or flux map, calculations at the saved points.

The restart file includes all the isotopic data needed to restart a calculation.
The restart file does not include the geometry description, so a regular input deck must also be used.
A user should set up an input deck for a fresh core, and then use the restart file to 
overwrite the fresh isotopic concentrations with the isotopic concentrations on the restart file.

A restart file can be written at any statepoint by using a {\it restart\_write} card,
\begin{verbatim}
  restart_write filename restart_label
\end{verbatim}
Where ``filename'' is the name of the restart file, and ``restart\_label'' is an arbitrary user label used
to differentiate multiple statepoints written to the same file.  Examples of restart labels
include ``100EFPD'', ``HZP'', ``22.56'', ``100EFPD\_ARO'', etc.
A restart file can include multiple statepoints as long as each one uses a different restart label.

If a {\it restart\_label} card is used with a {\it deplete} card, the restart file is written at the last
exposure step of the depletion.

In the following example, a depletion is performed and restart files are written at multiple statepoints.
\begin{verbatim}
  [STATE]
    deplete EFPD 0.0
    restart_write  restart_cyc12.h5 "BOC"
  [STATE]
    deplete EFPD 20 40 80 100
   ! restart file is written at last exposure step on deplete card
    restart_write  restart_cyc12.h5 "100EFPD"
  [STATE]
    deplete EFPD 150 200
    restart_write  restart_cyc12.h5 "200EFPD"
  [STATE]
    deplete EFPD 250 300
    restart_write  restart_cyc12.h5 "300EFPD"
  [STATE]
    deplete EFPD 350 400
    restart_write  restart_cyc12.h5 "400EFPD"
  [STATE]
    op_date 1994/05/23       ! include shutdown date for EOC
    power 80.0
    deplete EFPD 423.4
    restart_write  restart_cyc12.h5 "EFPD423_EOC"
\end{verbatim}

Another application of restart files is to write the final isotopic information at the end of cycle (EOC)
so the data can be shuffled to a new cycle.  (Core shuffles will be discussed in a later section.)
If writing a restart file at the EOC, the shutdown date should be included using the {\it op\_date} card.
The reason for including the shutdown date is so the code will be able to calculate the isotopic decay
during the outage.  An example of the {\it op\_date} card is shown in the last [STATE] block in the example above.


%%%%%%%%%%%%%%%%%%%%%%%%%%%%%%%%
\section{Reading Restart Files}

A restart file can be read by including a {\it restart\_read} card in the [STATE] block.
\begin{verbatim}
  restart_read filename restart_label
\end{verbatim}
where ``restart\_label'' is the label that was used to write the restart file.
The {\it restart\_read} card is used to restart an existing calculation, it is 
not used to do core shuffles.

In the following example, one of the restart files from the previous example
is read and a new calculation is performed with a different power and boron concentration.
\begin{verbatim}
  [STATE]
    power 50.0
    boron 800
    restart_read restart_cyc12.h5 "200EFPD"
\end{verbatim}

It is possible to write a statepoint in quarter-symmetry, then read the restart back in full-symmetry,
or vice-versa.

There is currently a restriction that a user should not include a {\it deplete} card in any 
[STATE] block where a restart is read.  Instead, the user should split the restart read and depletion
into separate blocks, like the following example shows:
\begin{verbatim}
  [STATE]
    restart_read restart_cycx.h5 "EFPD30"  ! read restart at 30 EFPD
  [STATE]
    deplete EFPD 60 90 
\end{verbatim}


%%%%%%%%%%%%%%%%%%%%%%%%%%%%%%%%
\section{Core Shuffling}

A core shuffle occurs when the fuel assemblies are rearranged in a core, and/or new fuel is added to the core.
Fuel assemblies can be brought in from the fuel pool that were discharged in previous cycles.
Fuel can even be added that was discharged from other units (cross-unit shuffle).

When performing a core shuffle, the user needs to specify where existing fuel assemblies were moved from, 
and what the new fuel assemblies look like.

When fuel isotopics are written to a restart file, the assembly locations are saved 
based assembly based on the {\it xlabel} and {\it ylabel} labels.
For example, with the following labels defined:
\begin{verbatim}
  [CORE]
    xlabel   R  P  N  M  L  K  J  H  G  F  E  D  C  B  A
    ylabel  01 02 03 04 05 06 07 08 09 10 11 12 13 14 15
\end{verbatim}
The assembly locations are defined as:
\begin{verbatim}
                     L-01 K-01 J-01 H-01 G-01 F-01 E-01
           N-02 M-02 L-02 K-02 J-02 H-02 G-02 F-02 E-02 D-02 C-02
      P-03 N-03 M-03 L-03 K-03 J-03 H-03 G-03 F-03 E-03 D-03 C-03 B-03
      P-04 N-04 M-04 L-04 K-04 J-04 H-04 G-04 F-04 E-04 D-04 C-04 B-04
 R-05 P-05 N-05 M-05 L-05 K-05 J-05 H-05 G-05 F-05 E-05 D-05 C-05 B-05 A-05
 R-06 P-06 N-06 M-06 L-06 K-06 J-06 H-06 G-06 F-06 E-06 D-06 C-06 B-06 A-06
 R-07 P-07 N-07 M-07 L-07 K-07 J-07 H-07 G-07 F-07 E-07 D-07 C-07 B-07 A-07
 R-08 P-08 N-08 M-08 L-08 K-08 J-08 H-08 G-08 F-08 E-08 D-08 C-08 B-08 A-08
 R-09 P-09 N-09 M-09 L-09 K-09 J-09 H-09 G-09 F-09 E-09 D-09 C-09 B-09 A-09
 R-10 P-10 N-10 M-10 L-10 K-10 J-10 H-10 G-10 F-10 E-10 D-10 C-10 B-10 A-10
 R-11 P-11 N-11 M-11 L-11 K-11 J-11 H-11 G-11 F-11 E-11 D-11 C-11 B-11 A-11
      P-12 N-12 M-12 L-12 K-12 J-12 H-12 G-12 F-12 E-12 D-12 C-12 B-12
      P-13 N-13 M-13 L-13 K-13 J-13 H-13 G-13 F-13 E-13 D-13 C-13 B-13
           N-14 M-14 L-14 K-14 J-14 H-14 G-14 F-14 E-14 D-14 C-14
                     L-15 K-15 J-15 H-15 G-15 F-15 E-15
\end{verbatim}
The restart files also include the cycle number (which is stored as a label), so
a cycle number and location can be used to define any assembly location in any cycle.
For example ``C3K-12'' refers to location ``K-12'' of cycle ``C3''.  If no cycle
number is specified, it defaults to the previous cycle number (i.e. cycle N-1).

New, fresh assemblies are defined by using a plus sign followed by the assembly type.
For example ``+ASMA'' signifies a fresh fuel assembly of type ``ASMA''.

Using these conventions, a new core loading pattern can be defined using a {\it shuffle\_label} map.
The {\it shuffle\_label} map is a core map showing the the previous assembly locations and
new assemblies fuel types.  

The following example is the full-core loading pattern for cycle 2 of the BEAVRS benchmark.
The cycle numbers are not used in the location labels because all of the assemblies
were moved from the previous cycle (cycle 1), and the default behavior is to use
the previous cycle number if no cycle is specified.
\begin{verbatim}
  [CORE]
    cycle C2
    op_date 1996/03/02     ! cycle startup date
  [STATE]
    shuffle_label
                       L-10 +X34 +X32 +X34 +X32 +X34 E-10
             G-10 +X32 +X32 L-02 P-12 N-03 B-12 E-02 +X32 +X32 J-10
        F-09 +X34 N-02 N-10 +X32 D-11 R-10 M-11 +X32 C-10 C-02 +X34 K-09
        +X32 P-03 L-08 +X32 M-09 E-15 G-08 L-15 D-09 +X32 H-05 B-03 +X32
   F-05 +X32 F-03 +X32 M-04 +X32 M-03 A-10 D-03 +X32 D-04 +X32 K-03 +X32 K-05
   +X34 P-05 +X32 G-04 +X32 N-08 R-09 G-14 A-09 H-03 +X32 J-04 +X32 B-05 +X34
   +X32 D-02 E-12 A-11 N-04 G-01 B-09 H-15 J-14 J-01 C-04 R-11 L-12 M-02 +X32
   +X34 N-13 F-15 H-07 F-01 B-07 A-08 F-14 R-08 P-09 K-15 H-09 K-01 C-03 +X34
   +X32 D-14 E-04 A-05 N-12 G-15 G-02 H-01 P-07 J-15 C-12 R-05 L-04 M-14 +X32
   +X34 P-11 +X32 G-12 +X32 H-13 R-07 J-02 A-07 C-08 +X32 J-12 +X32 B-11 +X34
   F-11 +X32 F-13 +X32 M-12 +X32 M-13 R-06 D-13 +X32 D-12 +X32 K-13 +X32 K-11
        +X32 P-13 H-11 +X32 M-07 E-01 J-08 L-01 D-07 +X32 E-08 B-13 +X32
        F-07 +X34 N-14 N-06 +X32 D-05 A-06 M-05 +X32 C-06 C-14 +X34 K-07
             G-06 +X32 +X32 L-14 P-04 C-13 B-04 E-14 +X32 +X32 J-06
                       L-06 +X34 +X32 +X34 +X32 +X34 E-06
\end{verbatim}

The next example shows a quarter-core shuffle map.  This map is not realistic,
but it shows how fresh assemblies are inserted along with assemblies from 
cycles 8, 19, 20, and 21.  The fresh assemblies all have fuel type ``A12''.
\begin{verbatim}
  [CORE]
    cycle 22               ! new cycle number
    op_date 2012/10/02     ! cycle startup date
  [STATE]
    shuffle_label
       8H-10  +A12  21E-03  +A12  21E-13 21G-02 21G-08 21N-04
       +A12  21O-08 20C-04 19L-07  +A12  21E-06  +A12  21K-03
      21C-11 20D-03 21E-08  +A12  21M-04  +A12  21K-04 21A-07
       +A12  19G-10  +A12  21P-08  +A12  21O-06 21B-06
      21O-11  +A12  21D-11  +A12  21B-07  +A12  21G-11
      21B-09 21F-05  +A12  21F-13  +A12  21L-06
      21H-09  +A12  21D-09 21F-02 21M-07
      21D-04 21C-09 21G-01
\end{verbatim}

At this time, there is a restriction where the user must also include an {\it assm\_map} card
in the input to specify the fresh fuel assemblies.  This restriction will be removed in the
future so that the fresh assembly types specified on the {\it shuffle\_label} card will be used.

In addition to the loading patterns, a list of restart files must be included to
define the restart search path.  The order of the restart files is important and they
must be in reverse chronological order.
\begin{verbatim}
  restart_shuffle
    restart_file_12.h5 EOC12
    restart_file_11.h5 EOC11
    restart_file_10.h5 EOC10
    restart_file_5.h5  EOC5
\end{verbatim}
The first restart file is used to define the ``previous'' cycle number.  The cycle number
from this file will be used as the default cycle number in the shuffle map.
The code will search for the assembly on the first file. If the assembly is not found,
the code will go to the second restart file, and so on.

The next section gives an example of a core shuffle.

%%%%%%%%%%%%%%%%%%%%%%%%%%%%%%%%%%%%%%%%%
\subsection{Core Shuffle Example}

Consider an example where a core shuffle is occuring at the beginning of cycle 3.
There are two EOC restart files that have already been written from cycles 1 and 2.

These examples are not complete, they only show the pertinant cards needed
to perform the core shuffle.

The EOC 1 restart file was generated with the following input:
\begin{verbatim}
  [CORE]
    cycle 1    ! could be any arbitrary string like CYC1, etc.
    xlabel   R  P  N  M  L  K  J  H  G  F  E  D  C  B  A
    ylabel  01 02 03 04 05 06 07 08 09 10 11 12 13 14 15
  [STATE]
    deplete EFPD ... 327.3   ! only last depletion date shown
    op_date "1993/03/01"     ! shutdown date
    restart_write restart_cyc1.h5 "EOC1"
  [ASSEMBLY]
    ! this input includes a definition of assembly type ASMA
\end{verbatim}

The EOC 2 restart file was generated with the following input:
\begin{verbatim}
  [CORE]
    cycle 2
    xlabel   R  P  N  M  L  K  J  H  G  F  E  D  C  B  A
    ylabel  01 02 03 04 05 06 07 08 09 10 11 12 13 14 15
  [STATE]
    deplete EFPD ... 426.3
    op_date "1994/03/05"    ! shutdown date cycle 2
    restart_write restart_cyc2.h5 "EOC_with_coastdown"
  [ASSEMBLY]
   ! this input includes a definition of assembly type ASMB
   ! but it must also include the description for ASMA from cycle 1
\end{verbatim}

The following input is used to shuffle to Cycle 3:
\begin{verbatim}
  [CORE]
    cycle 3
    op_date "1994/04/07"    ! start-up date of cycle
    xlabel   R  P  N  M  L  K  J  H  G  F  E  D  C  B  A
    ylabel  01 02 03 04 05 06 07 08 09 10 11 12 13 14 15

  [STATE]
    shuffle_label
     1H-10 +ASMC  E-03 +ASMC  E-13  G-02  G-08  N-04
     +ASMC  O-08  C-04  L-07 +ASMC  E-06 +ASMC  K-03
      C-11  D-03  E-08 +ASMC  M-04 +ASMC  K-04  A-07
     +ASMC  G-10 +ASMC  P-08 +ASMC  O-06  B-06
      O-11 +ASMC  D-11 +ASMC  B-07 +ASMC  G-11
      B-09  F-05 +ASMC  F-13 +ASMC  L-06
      H-09 +ASMC  D-09  F-02  M-07
      D-04  C-09  G-01

   ! One assembly was loaded from cycle 1 (in the center)
   ! This assembly had to have the cycle number prepended to it

   ! All of the other assemblies came from cycle 2.  This is the default cycle,
   ! and the cycle number did not have to be prepended.

   ! restart using the EOC restart files from cycles 1 and 2
    restart_shuffle
       restart_cyc2.h5 EOC_with_coastdown
       restart_cyc1.h5 EOC1

  [ASSEMBLY]
   ! include descriptions for ASMA, ASMB, ASMC if they are
   ! all used in cycle 3
\end{verbatim}

%%%%%%%%%%%%%%%%%%%%%%%%%%%%%%%%%
\subsection{Shutdown decay}
When performing a core shuffle, a shutdown decay is performed on each assembly to account for
the shutdown decay time.  The shutdown decay calculation is important for calculating the decay and
build-up of fission products, such as xenon and samarium.

The shutdown decay time is calculated using the shutdown date from when the assembly was discharged 
and the new cycle startup date.  The discharge date is the {\it op\_date} on the restart file
the assembly data was written.
The cycle start-up date is the {\it op\_date} in the core shuffle deck.

%%%%%%%%%%%%%%%%%%%%%%%%%%%%%%%%%
\subsection{Cross Unit Shuffle}

The shuffling methodology can support cross-unit shuffles.

To use cross-unit shuffling, the unit number must be specified in the [CORE] block.
\begin{verbatim}
  unit 1    ! unit 1 of a 2 unit site
\end{verbatim}

To read an assembly from a different unit, the unit label is prepended to the front of the location label
in the {\it shuffle\_label} card using a colon.
For example, ``U2:C3G-04'' is used to read the assembly from Unit ``U2'', cycle ``C3'' and location ``G-04''.

Once the location labels have been defined, the user can mix and match restart files from different units
in the {\it restart\_shuffle} card:
\begin{verbatim}
  restart_shuffle
    restart_file_U1_12.h5 EOC12
    restart_file_U2_5.h5  EOC
    restart_file_U1_11.h5 EOC11
    restart_file_U2_4.h5  EOC
    restart_file_U1_10.h5 EOC10
    restart_file_U2_3.h5  EOC
    restart_file_U1_5.h5  EOC5
\end{verbatim}
The only ``trick'' is to get this list in the right chronological order because an
assembly could theoretically go from U2:CYC3 to U1:CYC10 then back to U2:CYC6.  Therefore, the restarts
must be in the correct chronological order.  Remember that the cycle numbers are arbitrary strings,
so there really is no ``order'' to them.  The order is defined by the order specified in the
{\it restart\_shuffle} input.

The shutdown dates are written to each restart file so the shutdown decay will be correctly
calculated for each assembly.
It doesn't matter what unit the assembly came from, the right shutdown dates will be used.


