%*-------------------------------------------------------------------------*
%*                 Copyright 2013-2015 Core Physics, Inc.                  *
%*  Under terms of the contract to support CASL, there is a non-exclusive  *
%*  license for use of this work by or on behalf of the U.S. Government.   *
%*-------------------------------------------------------------------------*
%
% @version CVS $Id: materials.tex,v 1.22 2015/02/23 00:56:17 scott Exp $
%
%%%%%%%%%%%%%%%%%%%%%%%%%%%%%%%%%%%%%%%%%
\chapter{Materials}  \label{chap:materials}

This chapter contains a description of the material input.  There are two types of materials
in the input file -- structural materials (input with a {\it mat} card) and fuel materials
(input with a {\it fuel} card).

Structural materials can be defined in either the [CORE] block, or in the
geometry object blocks [ASSEMBLY], [INSERT], [CONTROL], and [DETECTOR].
If the materials are defined in the [CORE] block, they have global scope.
If the materials are defined in the geometry object blocks, they only have
scope in the block they are defined.
The reason for this is to maintain the modularity of the geometry objects.

Fuel materials can only be defined in [ASSEMBLY] blocks.

Materials are used in many different input cards.  They are used to define
cells, nozzles, core plates, baffles, grids, reflectors, etc.  Every material that
is used in the input must be defined with either a {\it mat} card or a {\it fuel} card.

%%%%%%%%%%%%%%%%%%%%%%%%%%%%%%%%
\section{Structural Materials}

Structural materials are not fuel and do not deplete.  Structural
materials are defined with the following input card:

{\bf mat} {\it user-mat} {\it density} ({\it library-name$_i$}, {\it frac$_i$}, i=1, I)

where:
\begin{itemize}
\item {\it user-mat} is a user-defined material name.  It is case sensitive.  {\it user-mat} is used to
define material names in other input cards such as {\it cell}, {\it grid}, {\it nozzle}, etc. (No default)
\item {\it density} is the material density in g/cc (No default)
\item {\it library-name} is a corresponding material library name(s) for the user material.  The library name
must be defined in the material composition file. (Default = {\it user-mat}).  Multiple library materials
can be mixed to form a single user material.
\item {\it frac}  is the fraction of the library material in the user material. (Default=1.0 if there is only one library material in the user material).
\end{itemize}

There are two special user materials,  ``mod'' and ``vacuum''.
The user can use these materials in cell definitions, but the code will automatically
determine the composition of these materials  based on T/H feedback and soluble boron concentrations.
The user is not allowed to define a user material named ``mod'' or ``vacuum'' on a {\it mat} card.

The material libraries are code-specific libraries.  Please refer to the specific physics codes to
determine which library materials are available.

Some example material cards are shown below.
\begin{verbatim}
  mat zirc4 6.56                   ! library-name defaults to user-name 
  mat zirx  6.56 zirc4 1.0         ! user-name does not equal to library-name
  mat B10   12.0 boron 1.0
  mat XYZ   6.0  zirc4 0.8 ss 0.2  ! define new mixture of 80% zirc4 and 20% ss
  mat ABCD  8.0  zirc4 0.8 ss 0.15 b4c 0.05
\end{verbatim}

%% Fixed bug Andrew found below ~Feb 2014, atomic and weight were switched
All of the material fractions must sum to either +1.0 or -1.0.
If positive fractions are used, the fractions refer to weight fractions.
If negative fractions are used, the fractions refer to atomic fractions.

%%%%%%%%%%%%%%%%%%%%%%
\subsection{Search Order}

Structural materials can be defined in either the [CORE] block or one of the
geometry object blocks.  When a material is referred to in a block, it will look
for the material definition in the following order:
\begin{enumerate}  % enumerate = numbered list
 \item The code will first look for the material name in the local block ([ASSEMBLY], [INSERT], [CONTROL], or [DETECTOR])
 \item If the material is not found in the local block, it will look in the [CORE] block
\end{enumerate}

If materials are defined in the [CORE] block, they have global scope over the entire input.
If materials are defined in other blocks, they only have scope over the local block.  This means
that two geometry object blocks can use different material definitions with the same name.
One example of this is that two assemblies can be defined with the material ``zirc'', but
the ``zirc'' can have different compositions in each of the assemblies.

%%%%%%%%%%%%%%%%%%%%%%%%%%
\section{Composition File}

When using Shift all library materials must be included in a material composition file.
The composition file is a code-specific file which contains a list of all
of the available library materials.

Examples of typical library materials that are present on the composition file include:
\begin{itemize}   % itemize = bullet list
  \item aic (silver-indium-cadmium)
  \item b4c (boron carbide)
  \item boron
  \item inc (Inconel)
  \item he  (helium)
  \item ss  (stainless steel)
  \item zirc4 (Zircaloy-4)
\end{itemize}
Refer to the code-specific composition file for a complete list of valid library materials.

%%%%%%%%%%%%%%%%%%%%%%%%
\section{Fuel Materials}

Fuel materials are defined with {\it fuel} cards.
Fuel materials are heavy metal oxides which are usually UO$_2$ with different U-235 enrichments.
Fuel materials may also include MOX fuel, which consists of mixtures of uranium, plutonium and other actinides.
Fuel materials are different from structural materials in that they deplete and have
additional properties as described below.

Fuel can only be defined in [ASSEMBLY] blocks, and fuel materials can only be referenced by {\it cell} cards
in the [ASSEMBLY] block they are defined in.

Fuel materials are defined with the following input card:

{\bf fuel} {\it user-mat} {\it density} {\it thden} / {\it U-235\_enrichment}
    \{{\it HM\_material$_i$}={\it HM\_enrichment$_i$}, i=1, N\}  \\
   \{ / {\it gad\_material}={\it gad\_fraction} \}

Where:

\begin{itemize}   % itemize = bullet list
  \item {\it user-mat} is a user-defined fuel name. It is case sensitive. (No default)
  \item {\it density} is the fuel material density in g/cc (No default).
     The density is used to calculate number densities.
  \item {\it thden} is the percent of theoretical density in the pellet (\%) (No default).
   The theoretical density is only used to look up material properties in the fuel performance,
   it is not used to calculate number densities.  There is no ``double counting'' between
   {\it density} and {\it thden}.
  \item {\it U-235\_enrichment} is the U-235 enrichment in the fuel in weight \%  (No default).
  \begin{itemize}   % itemize = bullet list
    \item If U-234 and U-236 are not specified, they will automatically be added to the
       fuel by a pre-determined function (see below)
    \item If the sum of the heavy metal (HM) enrichments does not equal 100\%, the remainder of the
        HM composition will be assigned to U-238.
  \end{itemize}
  \item {\it HM\_material$_i$} is the material name for HM isotope $i$ (Pu-239, Pu-241, etc.) (optional) 
    The names of the HM materials must be valid library-names.
  \item {\it HM\_enrichment$_i$} is the enrichment of HM isotope $i$ in weight \% (optional)
  \item {\it gad\_material} is the material name for the gadolina oxide (or other material) (optional).
        The gad material is usually a mixture defined on a separate {\it mat} card.
  \item {\it gad\_fraction} is the weight percent of the gad material relative to the total fuel mass (optional)
\end{itemize}

Oxygen should not be included on the {\it fuel} card.  The correct amount of oxygen will automatically
be added to the HM to create an oxide (either UO$_2$  or (HM)O$_2$).

The {\it density} is the ``stack density'' or ``smeared density'' and should include the volume
of the pellet dishing and chamfers.
It is calculated as the total mass of the fuel pellets divided by the total volume of the fuel
\begin{equation}
  \mbox{stack density} = \frac{ (\mbox{fuel mass})} { \pi (\mbox{pellet radius})^2 \, (\mbox{fuel height}) }
\end{equation}

The {\it thden} refers to the actual theoretical density of the pellet.  This quantity is
used in fuel performance codes to evaluate material properties.

If U-234 or U-236 enrichments are not included in the fuel definition, they are automatically
added to the fuel with the following formulas:
\begin{equation}
   W_{234} = 0.007731 \cdot W_{235}^{1.0837}
\end{equation}
\begin{equation}
  W_{236} = 0.0046 \cdot W_{235}
\end{equation}
Where $W_{23x}$ is the enrichment of each of the uranium isotopes in percent.

If a user specifically does NOT want U-234 or U-236, they should specify a U-234 
and/or U-236 enrichment of zero.

Examples of typical {\it fuel} cards are shown below.  The user
only has to specify the U-235 enrichment and the code will automatically add
U-234, U-236, U-238, and oxygen to the fuel.
\begin{verbatim}
  fuel U21 10.4 95.2 / 2.1        ! 2.1% enriched UO2 fuel, no gad
  fuel UO2-35 10.297 95.0 / 3.5   ! 3.5% enriched UO2 fuel, no gad
  fuel U23    10.111 / 2.3        ! fuel with default thden
\end{verbatim}

An example of a {\it fuel} card with gadolinia burnable poison is shown next.
In this example, the gadolinia oxide is first defined with a {\it mat} card and 
is mixed with the fuel as 5\%  gad oxide and 95\% UO$_2$ (weight percents). 
\begin{verbatim}
  mat gad5 7.407 gd2o3 1.0                ! define gad material separately
  fuel U49 10.111 94.5 / 1.8  / gad5=0.05 ! 1.8% enriched fuel with 5% gad
\end{verbatim}

Some examples of MOX fuel cards are shown next.  In these cards, the user specifies
the U-235 enrichment (the U-235 enrichment is usually small in MOX fuel) and the
plutonium isotope enrichments.  The code will automatically add U-234, U-236, U-238,
and oxygen.
\begin{verbatim}
  fuel PX1  10.111 94.5 / 0.711 Pu-239=23.4 Pu-241=1.9    ! Sample MOX fuel
  fuel MOX1 10.111 94.5 / 1.0   Pu-238=1.0  Pu-239=26.0 
                    Pu-240=12.0 Pu-241=8.0 Pu-242=3.0 
\end{verbatim}

Only oxide fuel can be defined on the {\it fuel} card. Metallic fuel is not supported.

