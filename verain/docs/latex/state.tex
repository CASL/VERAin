%*-------------------------------------------------------------------------*
%*                 Copyright 2013-2015 Core Physics, Inc.                  *
%*  Under terms of the contract to support CASL, there is a non-exclusive  *
%*  license for use of this work by or on behalf of the U.S. Government.   *
%*-------------------------------------------------------------------------*
%
% @version CVS $Id: state.tex,v 1.5 2015/02/17 03:26:30 Scott Exp $
%
%%%%%%%%%%%%%%%%%%%%%%%%%%%%%%%%%%%%%%%%%%%%%%%%%%%%%%%%%%%%%%%%
\section{State Description}

The [STATE] block defines the state of the core
(power, flow, pressure, inlet temperature, rod positions, boron concentration, etc.)
at a particular point in time.
These values will typically change during a cycle depletion.

An example showing the most common input cards in the [STATE] block is shown below.
A complete listing of all the input cards in the [STATE] block is located in Section~\ref{sec:statecards}.

\begin{verbatim}
  [STATE]
    power   98.0       ! % of rated power - rated values defined in [CORE] block
    flow   100.0       ! % of rated flow
    pressure 2250.0    ! psia
    tinlet 557.33 F    !
    feedback on        ! turn on T/H feedback

    boron   1285       ! initial boron ppmB
    search  boron      ! turn on boron search

    sym qtr            ! run problem in qtr-symmetry

    rodbank SA 228
            SB 228
            SC 228
            SD 228
             A 228
             B 228
             C 228
             D 167
\end{verbatim}

The {\it sym} card tells the code to run the calculation in full-core or
qtr-core symmetry. If the calculation is run in qtr-core symmetry, the symmetry is either
set to qtr-core rotational or qtr-core mirror by the {\it bc\_sym} card in the [CORE] block.

The {\it rodbank} card is used to position the control rods. The {\it rodbank} input includes pairs of bank names
and bank positions.  The bank names correspond to the {\it crd\_map} in the [CORE] block.
The positions indicate the position of the control rod bank in steps.
Step 0 is fully inserted.
The number of steps for a rod to be completely withdrawn is set by the {\it stroke} card in
the [CONTROL] block (see Section~\ref{sec:stroke}). For Westinghouse PWR's, a typical value of fully-withdrawn is 228 steps.

%%%%%%%%%%%%%%%%%%%%%%%%%%%%%%%
\section{Edits Description}

The [EDITS] block is used to control the output edits.

One of the edits produced by the core simulator is the rod power.   The user has 
the ability to specify the axial levels that the power is averaged over with 
the {\it axial\_edit\_bounds} card.
The user may choose to average power over uniform axial intervals (like most nodal codes),
or to specify the edit intervals manually.

(Note: the edit options are under development and more options will be added in the future.)

A complete listing of all the input cards in the [EDITS] block is located in Section~\ref{sec:editscards}.

\subsection{COBRA-TF Nodalization}

The {\it axial\_edit\_bounds} card is also used to set the axial nodalization when coupling
the neutronics physics code to the COBRA-TF (CTF) subchannel code.

When running CTF, there is a restriction that the grid boundaries must be explicitly
included in the {\it axial\_edit\_bounds}. This can get a little complicated for the user.
In the VERA input, spacer grids are defined in the [ASSEMBLY] block by specifying the 
grid heights on the {\it grid} card and the elevations of the grid midpoints 
on the {\it grid\_axial} card.
From the grid heights and midpoints, the elevations at the top and bottom of the spacer grid
can be calculated, and then the top and bottom elevations must be included 
in the {\it axial\_edit\_bounds}.

For example, if a grid is defined with a centerline at 75.0 and a height of 2.5, then the
{\it axial\_edit\_bounds} must include the points $75.0 \pm 1.25 = 73.75$ and $76.25$.

\begin{verbatim}
   [ASSEMBLY]
    grid GRID1 inc 1000 2.5    ! grid name, material, mass (g), and height (cm)
    grid_axial                 ! locations of grid midpoints (cm)
        GRID1 75.0

   [EDITS]
    axial_edit_bounds
       ...
       73.75           ! this array must include top and bottom grid boundaries
       76.25
       ...
\end{verbatim}

The reason for this restriction is because the power is already calculated on the {\it axial\_edit\_bounds},
so it is natural to use the same power distribution to couple to the CTF model as well.
The grids must be explicitly included in the CTF boundaries so the loss coefficients are calculated correctly.

In the future, this restriction may be lifted and an additional edit bounds array may
be added explicitly for CTF calculations.


%%%%%%%%%%%%%%%%%%%%%%%%%%%%%%%
\section{Coupling Description}

The [COUPLING] block defines the relaxation parameters and convergence criteria to be
used when coupling different physics codes.  These values are used to determine
convergence {\it between} physics codes.  Convergence criteria {\it within} a physics
code is controlled by the code-specific block.

Refer to Section~\ref{sec:couplingcards} for a complete listing of all the cards in the [COUPLING] block.

No code-specific information is included in the [COUPLING] block.  All code-specific information
is contained in the code-specific blocks.  The [COUPLING] block is only used to define
generic coupling parameters.

As an example, consider the following multi-physics code coupling:
\begin{enumerate}
  \item  Run T/H calculation
  \item  Run neutronics calculation
  \item  Check eigenvalue convergence \label{stepepsk}
  \item  Check power convergence \label{stepepsp}
  \item  Relax/dampen the power shape
  \item  If not converged, go to step 1.
\end{enumerate}
The eigenvalue convergence in step~\ref{stepepsk} uses the card {\it epsk} to check
the change in eigenvalue between coupled iterations.
There are additional eigenvalue convergence criteria {\it within} the neutronics
code, but the internal parameters are specified in the individual code blocks.

The power convergence in step~\ref{stepepsp} uses the card {\it epsp} to check
the change in power between coupled iterations.

Additional convergence checks are made on the peak fuel temperature, maximum change in density,
and change in boron concentration (if applicable).

The example shown above uses a Picard iteration to converge.  Picard iterations usually
need to apply a relaxation factor (also called a damping factor or under-relaxation factor) to
one or more of the calculated quantities to converge.
The relaxation factors are applied in the following manner:
\begin{equation}
  x = \omega \, x^{\mbox{new}} + (1 - \omega) x^{\mbox{old}}
\end{equation}
where $x$ is the calculated parameter and $\omega$ is the relaxation factor.
A relaxation factor of 1.0 signifies no relaxation is performed.
A relaxation factor $< 1.0$ signifies under-relaxation.

Relaxation factors can be specified for the point-wise power, point-wise temperature, and/or point-wise density.
The relaxation is applied to the transferred quantities sent between physics codes.  The state variables
within each physics code are not changed.

An example [COUPLING] input block is shown below.

\begin{verbatim}
   [COUPLING]
     epsk        5.0  ! eigenvalue convergence (pcm)
     eps_temp    1.0  ! temperature convergence (deg C)
     eps_boron   0.1  ! boron convergence (ppm)
     rlx_power   0.5  ! power relaxation factor
     rlx_tfuel   1.0  ! fuel temperature relaxation factor
     rlx_den     1.0  ! density relaxation factor
     maxiter     20   ! maximum number of coupled iterations
\end{verbatim}

A complete listing of all the input cards in the [COUPLING] block is located in Section~\ref{sec:couplingcards}.

